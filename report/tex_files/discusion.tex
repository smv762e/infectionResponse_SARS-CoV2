\section{Discusión}
Los datos presentados aquí sugieren que la respuesta al SARS-CoV-2 está desequilibrada con respecto al control de la replicación del virus frente a la activación de la respuesta inmunitaria adaptativa. Dada esta dinámica, los tratamientos para la COVID-19 tienen menos que ver con la respuesta del IFN y más con el control de la inflamación. Debido a que nuestros datos sugieren que numerosas quimiocinas e IL están elevadas en pacientes con COVID-19, los esfuerzos futuros deben centrarse en los medicamentos aprobados por la Administración de Drogas y Alimentos de los EE. UU. (FDA) que pueden implementarse rápidamente y tener propiedades inmunomoduladoras.

\begin{figure}[h]
	\fbox{\includegraphics[width=0.9\textwidth]{figures/independenceScale_meanConnectivity1.jpg}}
	\label{fig:sample_clustering}
	\fbox{\includegraphics[width=0.9\textwidth]{figures/independenceScale_meanConnectivity2.jpg}}
	\label{fig:sample_clustering}
\end{figure}
