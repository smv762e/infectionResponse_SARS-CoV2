\section{Introducción}
Los coronavirus son un grupo diverso de virus de ARN monocatenario de sentido positivo con una amplia gama de huéspedes vertebrados. Cuatro géneros comunes de coronavirus (alfa, beta, gamma y delta) circulan entre los vertebrados y causan enfermedades leves del tracto respiratorio superior en humanos y gastroenteritis en animales. Sin embargo, en las últimas dos décadas han surgido tres betacoronavirus humanos altamente patógenos a partir de eventos zoonóticos. En 2002-2003, el coronavirus 1 relacionado con el síndrome respiratorio agudo severo (SARS-CoV-1) infectó a $\approx$8000 personas en todo el mundo con una tasa de letalidad de $\approx$10\%, seguido por el coronavirus relacionado con el síndrome respiratorio de Oriente Medio (MERS-CoV), que ha infectado a $\approx$2500 personas con una tasa de letalidad de $\approx$36\% desde 2012. En la actualidad, el mundo sufre una pandemia de SARS-CoV-2, causante de la enfermedad por coronavirus 2019 (COVID-19) y tiene una tasa de mortalidad global que aún está por determinar.

La infección por SARS-CoV-2 se caracteriza por una variedad de síntomas que incluyen fiebre, tos y malestar general en la mayoría de los casos, pero en los casos más graves, pueden llegar a desarrollar un síndrome de dificultad respiratoria aguda y lesión pulmonar aguda, lo que provoca morbilidad y mortalidad causadas por daños en la luz alveolar que conducen a inflamación y neumonía.

Entendiendo qué respuesta puede tener el cuerpo cuando se infecta por SARS-CoV-2, ahora se estudiará esta respuesta, pero desarrollada en las células del epitelio del pulmón mediante el análisis de perfiles de expresión génica publicados en el dataset GEO GSE147507~\cite{BlancoMelo2020}.